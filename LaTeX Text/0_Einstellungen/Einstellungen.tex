%	2.) Einstellungen
%		a) Anzeigestil der Seiten														
%		b) Eigene Befehle																
%		c) Seiten-Layout	  
%		d) Trennungskorrekturen für automatischen Zeilenumbruch														  
%		e) Links im pdf-Dokument erstellen										  



%%%%%% -----------------------------------------------------------------
%%%%%% 2.) Einstellungen

%%% a) Anzeigestil der Seiten
\pagestyle{headings}														%% Verwendet den Standard-Seitenstil der Koma-Klasse
\renewcommand*{\chapterheadstartvskip}{\vspace*{.4\baselineskip}}			%% Verringert 



%%% b) Eigene Befehle
\newcommand{\abs}[1]{\ensuremath{\left\vert#1\right\vert}}					%% Fügt den Befehl \abs hinzu, der vertikale Striche um das Argument von \abs ergänzt



%%% c) Seiten-Layout
\oddsidemargin   0.0cm                           							%%
\evensidemargin  0.0cm                           							%%
\topmargin      -1.0cm         												%%
\textheight     24.0cm         												%%
\textwidth      16.0cm         												%%


\sloppy


%%% d) Trennungskorrekturen für automatischen Zeilenumbruch
\hyphenation{Glei-chung}													%% Zeigt Latex, dass das angegebene Wort an den Stellen mit "-" beim Zeilenumbruch geteilt werden darf
\hyphenation{Geo-me-trie}
\hyphenation{Mo-dell}
\hyphenation{Profil-änderung}
\hyphenation{zeit-genaue}
\hyphenation{Um-ge-bung}
\hyphenation{Kraft-stoff}
\hyphenation{Marge}
\hyphenation{üblicher-weise}
\hyphenation{Verdränger-pumpe}
\hyphenation{Sättigungs-dampf-druck}
\hyphenation{gewähr-leisten}
\hyphenation{höherem}
\hyphenation{gehört}
\hyphenation{genutzt}
\hyphenation{purem}
\hyphenation{finden}
\hyphenation{beträgt}
\hyphenation{jedoch}
\hyphenation{eine}
\hyphenation{paral-lelen}
\hyphenation{Familie}
\hyphenation{Leit-ungen}
\hyphenation{betrach-teten}
\hyphenation{Archi-tektur}
\hyphenation{Archi-tekturen}
\hyphenation{Kraft-stoff-system}
\hyphenation{Kraft-stoff-systeme}
\hyphenation{Ein-tritts-temperaturen}
\hyphenation{Ein-tritts-temperatur}
\hyphenation{Lösungs-Algo-rithmus}
\hyphenation{GasTurb}
\hyphenation{einer}
\hyphenation{ent-scheidet}
\hyphenation{Kraft-stoff-eigen-schaften}
\hyphenation{physik-basiert}
\hyphenation{physik-basierte}
\hyphenation{physik-basierter}
\hyphenation{physik-basierten}
\hyphenation{ins-be-sondere}
\hyphenation{Reise-flugs}
\hyphenation{Reise-flug}
\hyphenation{Druck-ver-luste}
\hyphenation{Nieder-druck-system}
\hyphenation{Gas-turbinen}
\hyphenation{Mehr-bedarf}
\hyphenation{andere}
\hyphenation{Wärme-ein-träge}
\hyphenation{Druck-er-höhung}
\hyphenation{Model-lierung}
\hyphenation{Inter-aktion}
\hyphenation{Inter-aktionen}
\hyphenation{Lei-tun-gen}
\hyphenation{dieser}
\hyphenation{haben}
\hyphenation{Daten-grund-lage}
\hyphenation{Ab-häng-ig-keit}
\hyphenation{Para-meter}
\hyphenation{Nieder-druck-pumpe}
\hyphenation{Hoch-druck-pumpe}
\hyphenation{Take-off}
\hyphenation{Krei-sel-pumpe}
\hyphenation{Krei-sel-pumpen}
\hyphenation{Be-triebs-punkt}
\hyphenation{Re-zir-ku-la-tions-ver-dich-ter}

\global\righthyphenmin=10
\global\lefthyphenmin=10


%%% e) Links im pdf-Dokument erstellen
\hypersetup{
	urlcolor = {black},
	plainpages = {false}, 													%% Verwendet nur arabische Seiten-Label
	breaklinks = {true},													%% Erlaubt Zeilenumbrüche in Links
	colorlinks = {true},													%% Benutze farbige Links
	linkcolor = {black}, 													%% Farbe der Links im Dokument
	citecolor = {black},													%% Linkfarbe im PDF schwarz machen. Für elektronische Version ändern.
	bookmarksnumbered = {true}, 											%% Einträge im Verzeichnis nummerieren
	pdftitle = {Abschlussarbeit Vorname Nachmane}, 							%% Titel der Arbeit
	pdfsubject = {Titel der Abschlussarbeit},								%% Thema der Arbeit
	pdfauthor = {Vorname Nachname},											%% Autor der Arbeit
	%pdfkeywords = {}, 														%% Stichwörter zur Arbeit
	pdfproducer = {pdfLatex}, 												%% Erzeugt durch
	pdfcreator = {LaTeX (MikTex mit TexnicCenter)} 							%% Erstellt mit
}

%%% Darstellung von Einheiten
\sisetup{per-mode=symbol}
\sisetup{group-separator = {.}}
\sisetup{output-decimal-marker = {,}}

\makeatletter
\newcommand*{\rom}[1]{\expandafter\@slowromancap\romannumeral #1@}
\makeatother