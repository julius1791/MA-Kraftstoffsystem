\chapter*{Symbolverzeichnis}
\addcontentsline{toc}{chapter}{Symbolverzeichnis}
\markboth{Symbolverzeichnis}{Symbolverzeichnis}

%%%%%%%%%%%%%%%%%%%%%%%%%%%%%%%%%%%%%%%%%%%%%%%%%%%%%%%%%%%%%
%%%%% Lateinische Formelzeichen
\section*{Lateinische Formelzeichen}

\begin{tabbing}
	\hspace*{3cm} \= \hspace*{8cm} \= \hspace*{2cm}\kill
	\textbf{Zeichen} \> \textbf{Bedeutung} 				\>	\textbf{Einheit}		\\[5mm]
	$a$         \>  spezifische Helmholtz-Energie       \> \si{\J\per\kg}	 \\
    $c_p$       \>  spezifische isobare Wärmekapazität  \>  \si{\J\per\kg\per\K}	 \\
    $c_v$       \>  spezifische isochore Wärmekapazität \>  \si{\J\per\kg\per\K} \\
    $D$         \>  Durchmesser                         \>  \si{\m}  \\
    $F$         \>  Schub                               \>  \si{\N}  \\
    $h$         \>  spezifische Enthalpie               \>  \si{\J\per\kg} \\
    $H$         \>  Flughöhe                            \>  \si{\m} \\
    $H_u$       \>  unterer Heizwert                    \>  \si{\J\per\kg} \\
    $\dot{H}$   \>  Enthalpiestrom                      \>  \si{\W}   \\
    $L$         \>  Länge                               \>  \si{\m}   \\
    $M_R$       \>  molare Masse                        \>  \si{\kg\per\kmol} \\
    $Ma$        \>  Machzahl                            \>  -   \\
    $\dot{m}$   \>  Massenstrom                         \>  \si{\kg\per\s} \\
    $n$         \>  Iteration                           \>  -  \\
    $N$         \>  Drehzahl                            \>  \si{1\per\min} \\
    $p$         \>  Druck                               \>  \si{\Pa}  \\
    $P$         \>  Parameter                           \>   \\
    $P$         \>  Leistung                            \>  \si{\W}   \\
    $q$         \>  spezifische Wärme                   \>  \si{\J\per\kg} \\
    $\dot{Q}$   \>  Wärmestrom                          \>  \si{\W}   \\
    $R$         \>  spezifische Gaskonstante            \>  \si{\J\per\kg\per\K} \\
    $s$         \>  spezifische Entropie                \>  \si{\J\per\kg\per\K} \\
    $T$         \>  Temperatur                          \>  \si{\K}   \\ 
    $v$         \>  spezifisches Volumen                \>  \si{\m\cubed\per\kg} \\
    $v$         \>  Geschwindigkeit                     \>  \si{\m\per\s} \\
    $w$         \>  Massenanteil                        \>  -   \\
\end{tabbing}

%%%%%%%%%%%%%%%%%%%%%%%%%%%%%%%%%%%%%%%%%%%%%%%%%%%%%%%%%%%%%
%%%%% Griechische Formelzeichen
\section*{Griechische Formelzeichen}

\begin{tabbing}
	\hspace*{3cm} \= \hspace*{8cm} \= \hspace*{2cm}\kill
	\textbf{Zeichen} \> \textbf{Bedeutung} 				\>	\textbf{Einheit}		\\[5mm]
	$\alpha$    \>  entdimensionierte Helmholtz-Energie \>  -  \\
    $\delta$    \>  entdimensionierte Dichte            \>  -  \\
    $\eta$		\>	Wirkungsgrad						\>	-  \\
	$\kappa$	\>	Isentropenexponent					\>	-  \\
    $\lambda$   \>  Rohrreibungswert                    \>  -  \\
    $\phi$      \>  Kraftstoff-Luft-Äquivalenzverhältnis \> -  \\
	$\pi$		\>	Druckverhältnis						\>	-  \\
	$\rho$	    \>	Dichte								\>     \si{\kg\per\cubic\m}	\\
    $\tau$      \>  entdimensionierte Temperatur        \>  -  \\
\end{tabbing}

%%%%%%%%%%%%%%%%%%%%%%%%%%%%%%%%%%%%%%%%%%%%%%%%%%%%%%%%%%%%%
%%%%% Indizes
\section*{Indizes}

\begin{tabbing}
	\hspace*{3cm} \= \hspace*{8cm} \kill
	\textbf{Zeichen} \> \textbf{Bedeutung} 							\\[5mm]
	0		\>	Eintritt in das Kraftstoffsystem					\\
    0		\>	nicht korrigiert					                \\
    0       \>  Referenz 						                    \\
    0       \>  Idealgas                                            \\
	1		\>	Ausgangszustand                                     \\
    1       \>  Niederdruckwelle                                    \\
    \rom{1} \> Fluid 1                                              \\
	2		\>	Endzustand								            \\
    2       \>  Hochdruckwelle                                      \\
    \rom{2} \> Fluid 2                                              \\
    B       \>  Abgas                                               \\
    BK      \>  Brennkammer                                         \\
    $c$     \>  kritischer Punkt                                    \\
    F       \>  Fan                                                 \\
    FOHE    \>  (Haupt-) Ölsystem-Wärmeübertrager                   \\
    gsmt    \>  Kraftstoffsystem                                    \\
    h       \>  hoher Druck                                         \\
    H       \>  hohe Temperatur                                     \\
    H$_2$   \>  Wasserstoff                                         \\
    H$_2$O  \>  Wasser                                              \\
    HP      \>  Hochdruck                                           \\
    HPFC    \>  Hochdruckverdichter                                 \\
    HPFP    \>  Hochdruckpumpe                                      \\
    IDG     \>  Stromgenerator-Ölsystem-Wärmeübertrager             \\
    inj     \>  Injektor                                            \\
    ISA     \>  Normatmosphäre                                      \\
    Jet-A   \>  Kerosin                                             \\
    k       \>  Kraftstoff                                          \\
    L       \>  Leitung                                             \\
    L       \>  Luft                                                \\
    LP      \>  Niederdruck                                         \\
    LPFP    \>  Niederdruckpumpe                                    \\
    mix     \>  Kraftstoffmischung                                  \\
    MTO     \>  Startfall                                           \\
    n       \>  niedriger Druck                                     \\
    N       \>  niedrige Temperatur                                 \\
    N$_2$   \>  Stickstoff                                          \\
    O$_2$   \>  Sauerstoff                                          \\
    P       \>  Leistungsentnahme                                   \\
    PHC     \>  parallele Wasserstoffverbrennung                    \\
    r       \>  Kompressibilität                                    \\
    R       \>  rezirkuliert                                        \\
    ref     \>  Referenz                                            \\
    RV      \>  Rezirkulationsverdichter                            \\
    $s$     \>  isentrop                                            \\
    st      \>  stöchiometrisch                                     \\
    $t$     \>  Totalzustand                                        \\
    U       \>  Umgebungszustand                                    \\
    V       \>  Verdampfer                                          \\
    V       \>  Vormischung                                         \\
    W       \>  Wärmeübertrager                                     \\
    Z       \>  Zapfluft                                            \\

\end{tabbing}

%%%%%%%%%%%%%%%%%%%%%%%%%%%%%%%%%%%%%%%%%%%%%%%%%%%%%%%%%%%%%
%%%%% Abkürzungen
\section*{Abkürzungen}

\begin{tabbing}
	\hspace*{3cm} \= \hspace*{8cm} \kill
	\textbf{Zeichen} \> \textbf{Bedeutung} 							\\[5mm]
    ADP     \>  Advanced Ducted Propeller                           \\
    EEC     \>  engl.: Electronic Engine Control                    \\
    ECS     \>  engl.: Environmental Control System                 \\
    FDGS    \>  engl.: Fan Drive Gear System                        \\
    FMU     \>  engl.: Fuel Metering Unit                           \\
    FOHE    \>  engl.: Fuel Oil Heat Exchanger                      \\
    FRV     \>  engl.: Fuel Return Valve                            \\
    H$_2$ 	\> 	Wasserstoff 										\\
    HMU     \>  engl.: Hydromechanical Unit                         \\
    HPFC    \>  engl.: High-Pressure Fuel Compressor                \\
    HPFP    \>  engl.: High-Pressure Fuel Pump                      \\
    LPFP    \>  engl.: Low-Pressure Fuel Pump                       \\
    MTO     \>  engl.: Max Takeoff                                  \\
    IDG     \>  engl.: Integrated Drive Generator                   \\
    ISA     \>  Internationale Standardatmosphäre                   \\
	IST		\> 	Institut für Strahlantriebe und Turbomaschinen		\\
    PHC     \>  engl.: Parallel Hydrogen Combustion                 \\
    PHCHE   \>  engl.: PHC Heat Exchanger                           \\
    RV      \>  Rezirkulationsverdichter                            \\
    SAF 	\> 	engl.: Sustainable Aviation Fuels 					\\
	VDI		\>	Verein Deutscher Ingenieure							\\

\end{tabbing}