%==============================================================================
\chapter{Einleitung}\label{chap:einleitung}
%==============================================================================


Die Verbrennung fossiler Kraftstoffe führt zur Emission von Treibhausgasen, die die Klimaerwärmung vorantreiben und damit die Existenzgrundlagen der Menschheit bedrohen \cite{UN_ClimateChange}. Angesichts dieser Gefahr und der zunehmenden Häufigkeit extremer Wetterereignisse verpflichten sich immer mehr Staaten durch multilaterale Abkommen Maßnahmen gegen die globale Erwärmung zu ergreifen \cite{paris}. Etwa $5\,\%$ des globalen Strahlungsantriebs ist auf die zivile Luftfahrt zurückzuführen \cite{Grobler2024}. Reduktionen der Treibhausgasemissionen sind daher auch in der Luftfahrtindustrie zwingend erforderlich. Im Rahmen der „Flight Path 2050“ Initiative fordert die Europäische Kommission eine Reduktion der spezifischen Emission von Kohlenstoffdioxid-Äquivalenten von $75\,\%$ bis 2050 \cite{EuropeanCommission2011}.

Angesichts des prognostizierten zunehmenden Verkehrsaufkommens werden evolutionäre Weiterentwicklungen konventioneller Flugantriebe und die Nutzung von nachhaltigen Kraftstoffen (engl. Sustainable Aviation Fuels, SAF) allein wahrscheinlich nicht ausreichen, um die ehrgeizigen Klimaziele zu erreichen \cite{cleanskies}. Daher sind neuartige, auch als revolutionär bezeichnete Antriebskonzepte erforderlich. Wasserstoffbetriebene Antriebe könnten eine zentrale Rolle bei der Dekarbonisierung der Luftfahrt spielen. Die Nutzung von Wasserstoff (H$_2$) verursacht keine lokalen Kohlenstoffdioxidemissionen. Zudem kann Wasserstoff klimaneutral produziert werden und könnte  ökonomische Vorteile im Vergleich zu mit SAF betriebenen Flugantrieben bieten \cite{VanLandingham}.

Neben Brennstoffzellen kann Wasserstoff für Luftfahrtanwendungen auch in Gasturbinen genutzt werden. Zwar sind wasserstoffbetriebene Gasturbinentriebwerke im Gegensatz zu Brennstoffzellenantrieben nicht vollständig lokal Emissionsfrei, jedoch weisen sie eine höhere Leistungsdichte auf. Zudem stellt die Zertifizierung wasserstoffbetriebener Gasturbinen aufgrund ihrer größeren Ähnlichkeit mit kerosinbetriebenen Flugantrieben eine geringere Hürde dar. \cite{Kadyk.2018}

Kerosinbetriebenen und wasserstoffbetriebenen Gasturbinentriebwerke unterscheiden sich insbesondere in ihren Kraftstoffsystemen. Um höhere volumetrische Energiedichten und damit größere Reichweiten zu ermöglichen wird Wasserstoff in Verkehrsflugzeugen im flüssigen Zustand und somit bei kryogenen Temperaturen gelagert. Vor dem Eintritt in die Brennkammer muss der Wasserstoff im Kraftstoffsystem verdampft und erhitzt werden. Dies stellt einen erheblichem energetischen Mehraufwand gegenüber kerosinbetriebenen Kraftstoffsystemen dar. 

Ziel dieser Arbeit ist die Entwicklung einer Methodik für die Modellierung von Kraftstoffsystemen für Fluggasturbinen. Im Fokus liegt dabei die Schaffung von Vergleichbarkeit der Wärme- und Leistungsbedarfe zwischen verschiedenen Kraftstoffsystemen. Zunächst werden die Grundlagen von kerosin- und wasserstoffbetriebener Kraftstoffsysteme erläutert. Anschließend werden die für die Modellierung von Kraftstoffsystemen notwendige Komponenten identifiziert und zu einem kerosinbetriebenen sowie unterschiedlichen wasserstoffbetriebenen Kraftstoffsystemen zusammengesetzt. Abschließend werden mithilfe einer Sensitivitätsanalyse relevante Auslegungsparameter identifiziert und im Rahmen einer Parameterstudie das Systemverhalten untersucht.