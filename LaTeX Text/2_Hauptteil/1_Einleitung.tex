%==============================================================================
\chapter{Einleitung}\label{chap:einleitung}
%==============================================================================


%Die Verbrennung fossiler Kraftstoffe führt zur Emission von Treibhausgasen, die die Klimaerwärmung vorantreiben und damit die Existenzgrundlagen der Menschheit bedrohen \cite{UN_ClimateChange}. Angesichts dieser Gefahr und der zunehmenden Häufigkeit extremer Wetterereignisse verpflichten sich immer mehr Staaten durch multilaterale Abkommen Maßnahmen gegen die globale Erwärmung zu ergreifen \cite{paris}. Etwa $5\,\%$ des globalen Strahlungsantriebs ist auf die zivile Luftfahrt zurückzuführen \cite{Grobler2024}. Reduktionen der Treibhausgasemissionen sind daher auch in der Luftfahrtindustrie zwingend erforderlich. Im Rahmen der „Flight Path 2050“ Initiative fordert die Europäische Kommission eine Reduktion der spezifischen Emission von Kohlenstoffdioxid-Äquivalenten von $75\,\%$ bis 2050 \cite{EuropeanCommission2011}.

Um die ambitionierten Klimaziele zu erreichen, werden neben der evolutionären Weiterentwicklung kerosinbetriebener Flugantriebe und der Nutzung nachhaltiger Kraftstoffe (engl. Sustainable Aviation Fuels, SAF) auch revolutionäre Antriebskonzepte erforscht \cite{cleanskies, EuropeanCommission2011}. Eine vielversprechende Technologie in diesem Bereich sind wasserstoffbetriebene Antriebe, da die Nutzung dieses Kraftstoffs keine lokalen Kohlenstoffdioxidemissionen verursacht. Zudem kann Wasserstoff klimaneutral produziert werden und könnte  ökonomische Vorteile im Vergleich zu mit SAF betriebenen Flugantrieben bieten \cite{VanLandingham}.

Neben Brennstoffzellen kann Wasserstoff für Luftfahrtanwendungen auch in Gasturbinen genutzt werden. Zwar sind wasserstoffbetriebene Gasturbinentriebwerke im Gegensatz zu Brennstoffzellen-Antrieben nicht vollständig lokal Emissionsfrei, jedoch weisen sie eine höhere Leistungsdichte auf. Zudem stellt die Zertifizierung wasserstoffbetriebener Gasturbinen aufgrund ihrer größeren Ähnlichkeit mit kerosinbetriebenen Flugantrieben eine geringere Hürde dar im Vergleich zu Brennstoffzellen-Antrieben. \cite{Kadyk.2018}

Kerosin- und wasserstoffbetriebene Gasturbinentriebwerke unterscheiden sich insbesondere in ihren Kraftstoffsystemen. Während Kerosin bei vergleichsweise hohen Temperaturen gelagert und in flüssiger Form in die Brennkammer eingespritzt wird, muss Flüssigwasserstoff bei niedrigen, kryogenen Temperaturen gelagert werden und wird vor der Einspritzung in die Brennkammer verdampft. Der resultierende erhebliche Mehrbedarf an Energie muss dem Kraftstoff in den Wärmeübertragern des Kraftstoffsystems zugeführt werden. Zudem müssen die Wasserstoffpumpen aufgrund der geringeren Dichte von Wasserstoff im Vergleich zu Kerosin eine höhere spezifische Arbeit leisten, um denselben Förderdruck zu erreichen. 

Ziel dieser Arbeit ist die Entwicklung einer Methodik für die Modellierung von Kraftstoffsystemen für Fluggasturbinen. Der Schwerpunkt liegt auf der Vergleichbarkeit der Wärme- und Leistungsbedarfe verschiedener Kraftstoffsysteme in erster Größenordnung, um deren unterschiedliche Anforderungen in der Triebwerks-Leistungsrechnung zu berücksichtigen. Hierfür wird eine wenig komplexe Modellierung mit einer möglichst geringen Anzahl an Modellannahmen angestrebt.

Zunächst werden die Grundlagen von Kerosin- und Wasserstoff-Kraftstoffsystemen erläutert. Anschließend werden die für die Modellierung von Kraftstoffsystemen notwendigen Komponenten identifiziert und zu einem Kerosin- sowie mehreren Wasserstoff-Kraftstoffsystemen zusammengesetzt. Die vollendeten Modellierungen werden im Rahmen einer Sensitivitätsanalyse auf relevante Auslegungsparameter untersucht und die Methodik anhand einer bestehenden Betrachtung validiert. Abschließend wird das Systemverhalten der modellierten Wasserstoff-Kraftstoffsysteme in einer Parameterstudie analysiert.