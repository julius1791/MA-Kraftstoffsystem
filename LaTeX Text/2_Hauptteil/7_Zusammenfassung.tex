%==============================================================================
\chapter{Zusammenfassung und Ausblick}
\label{chap:fazit}
%==============================================================================

Der Fokus dieser Arbeit ist die Entwicklung einer Modellierung zur Abschätzung dieser Anforderungen, insbesondere in Abhängigkeit der durch die Systemauslegung zu bestimmenden Eintrittstemperaturen. Im Rahmen dieser Arbeit wurden ein Kerosin-Kraftstoffsystem und drei Wasserstoff-Kraftstoffsysteme entwickelt. Die Modellierung der Kraftstoffsysteme erfolgte anhand von Komponenten- und Stoffmodellen, die auf Basis einer umfassenden Literaturrecherche für den Betriebspunkt eines Verkehrsflugzeugs im Reiseflug parametriert wurden. Die zugrunde liegenden Komponenten- und Stoffmodelle können flexible in unterschiedlichen Anordnungen kombiniert werden, um alternative Kraftstoffsystem-Konzepte abzubilden und zu vergleichen.

Die Methodik dieser Arbeit wurde auf ein bekanntes Kraftstoffsystem aus der Literatur angewandt, um die Aussagekraft des Ansatzes zu bestätigen. Das Ergebnis dieser Validierung ergab nur eine eingeschränkte Übereinstimmung mit den Literaturdaten. Der Ursprung dieser Abweichung ist nicht abschließend geklärt. Im Rahmen einer Sensitivitätsanalyse wurden die Wirkungsgrade der Verdichter und die Druckverluste im Hochdrucksystem der Wasserstoff-Kraftstoffsysteme als einflussreiche Parameter identifiziert. Auf das Kerosin-Kraftstoffsystem haben insbesondere der Wirkungsgrad der Hochdruckpumpe und der Niederdruckpumpen-Austrittsdruck einen erheblichen Einfluss auf die Modellierung der Gesamtleistung. 

Im Rahmen einer zweidimensionalen Parameterstudie der Wasserstoff-Kraftstoffsysteme wurde ein erheblicher Einfluss der Differenz zwischen Wärmeübertrager- und Brennkammer-Eintrittstemperatur auf den Leistungsbedarf der Wasserstoff-Kraftstoffsysteme festgestellt. Der Wärmebedarf der Wasserstoff-Kraftstoffsysteme ist weitestgehend proportional zur Brennkammer-Eintrittstemperatur. Ein Vergleich mit dem Kerosin-Kraftstoffsystem ergibt einen bis zu 30-Fach höheren Leistungsbedarf der Wasserstoff-Kraftstoffsysteme. Zudem weisen die Wasserstoff-Kraftstoffsysteme einen Wärmefehlbetrag von bis zu \SI{280}{\kilo\W} auf.

Zukünftige Arbeiten könnten neben abweichenden Anordnungen der Komponenten der Wasserstoff-Kraftstoffsysteme weitere Betriebspunkte untersuchen. Fortgeschrittene Arbeiten könnten sich damit beschäftigen die Modellierung auf Gesamttriebwerksebene in Leistungsrechnungen zu integrieren und die Modellierung durch die Berücksichtigung von transienten Betriebszuständen zu erweitern. Dies würde die Entwicklung einer Regelstrategie für die einzelnen Kraftstoffsystemkomponenten erfordern.