%	2.) Einstellungen
%		a) Anzeigestil der Seiten														
%		b) Eigene Befehle																
%		c) Seiten-Layout	  
%		d) Trennungskorrekturen für automatischen Zeilenumbruch														  
%		e) Links im pdf-Dokument erstellen										  



%%%%%% -----------------------------------------------------------------
%%%%%% 2.) Einstellungen

%%% a) Anzeigestil der Seiten
\pagestyle{headings}														%% Verwendet den Standard-Seitenstil der Koma-Klasse
\renewcommand*{\chapterheadstartvskip}{\vspace*{.4\baselineskip}}			%% Verringert 



%%% b) Eigene Befehle
\newcommand{\abs}[1]{\ensuremath{\left\vert#1\right\vert}}					%% Fügt den Befehl \abs hinzu, der vertikale Striche um das Argument von \abs ergänzt



%%% c) Seiten-Layout
\oddsidemargin   0.0cm                           							%%
\evensidemargin  0.0cm                           							%%
\topmargin      -1.0cm         												%%
\textheight     24.0cm         												%%
\textwidth      16.0cm         												%%


\sloppy


%%% d) Trennungskorrekturen für automatischen Zeilenumbruch
\hyphenation{Glei-chung}													%% Zeigt Latex, dass das angegebene Wort an den Stellen mit "-" beim Zeilenumbruch geteilt werden darf
\hyphenation{Geo-me-trie}
\hyphenation{Mo-dell}
\hyphenation{Profil-änderung}
\hyphenation{zeit-genaue}
\hyphenation{Um-ge-bung}
\hyphenation{Kraft-stoff}
\hyphenation{Marge}
\hyphenation{üblicher-weise}
\hyphenation{Verdränger-pumpe}
\hyphenation{Sättigungs-dampf-druck}
\hyphenation{gewähr-leisten}
\hyphenation{höherem}
\hyphenation{gehört}
\hyphenation{Kraftstoff-system}
\hyphenation{Kraftstoff-systeme}
\hyphenation{Eintritts-temperaturen}
\hyphenation{Eintritts-temperatur}
\hyphenation{GasTurb}
\hyphenation{einer}
\hyphenation{Nieder-druck-system Nieder-druck-systems erfolgt sonstiger abge-kühlt Schmal-rumpf-verkehrs-flugzeuge}

\global\righthyphenmin=10
\global\lefthyphenmin=10


%%% e) Links im pdf-Dokument erstellen
\hypersetup{
	urlcolor = {black},
	plainpages = {false}, 													%% Verwendet nur arabische Seiten-Label
	breaklinks = {true},													%% Erlaubt Zeilenumbrüche in Links
	colorlinks = {true},													%% Benutze farbige Links
	linkcolor = {black}, 													%% Farbe der Links im Dokument
	citecolor = {black},													%% Linkfarbe im PDF schwarz machen. Für elektronische Version ändern.
	bookmarksnumbered = {true}, 											%% Einträge im Verzeichnis nummerieren
	pdftitle = {Abschlussarbeit Vorname Nachmane}, 							%% Titel der Arbeit
	pdfsubject = {Titel der Abschlussarbeit},								%% Thema der Arbeit
	pdfauthor = {Vorname Nachname},											%% Autor der Arbeit
	%pdfkeywords = {}, 														%% Stichwörter zur Arbeit
	pdfproducer = {pdfLatex}, 												%% Erzeugt durch
	pdfcreator = {LaTeX (MikTex mit TexnicCenter)} 							%% Erstellt mit
}

%%% Darstellung von Einheiten
\sisetup{per-mode=symbol}
\sisetup{group-separator = {.}}
\sisetup{output-decimal-marker = {,}}

\makeatletter
\newcommand*{\rom}[1]{\expandafter\@slowromancap\romannumeral #1@}
\makeatother