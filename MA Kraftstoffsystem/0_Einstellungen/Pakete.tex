%	1.) Pakete	
%		a) Sprache, Zeichen, Symbole	
%		b) Bilder	
%		c) Tabellen	
%		d) Formatierung
%		e) Sonstige	


%%%%%% -----------------------------------------------------------------
%%%%%% 1.) Pakete

%%% a) Sprache, Zeichen, Symbole
\usepackage[ngerman]{babel}					             					%% BABEL -> deutsch | letztgenannte Sprache ist Hauptsprache im Dokument;Sprachwechsel im Dokument: \selectlanguage{ngerman}
\usepackage[utf8]{inputenc}                      							%% ISO-Text mit Umlauten
\usepackage[T1]{fontenc}                         							%% Zeichensatz mit Umlauten
\usepackage{amssymb}                            							%% Symbole und Umgebungen aus AMSTeX
\usepackage{amsmath} 														%% AMS-Mathematik
\usepackage{textcomp}														%% Zusätzliche Symbolzeichen, u.a. für Copyleft Zeichen
\usepackage{eurosym} 														%% Eurosymbol einfügen
\usepackage{bbding}															%% Zusätzliche Sonderzeichen
\usepackage{xfrac}															%% Andere Darstellung für Brüche. Befehl: \sfrac{}{}
\usepackage{lmodern}														%% Verhindert verpixelte Schrift und sorgt für eine bessere Darstellung von Schrift mit inputenc
\usepackage{siunitx}														%% Bessere Darstellung von Einheiten
% \usepackage[none]{hyphenat}



%%% b) Bilder
\usepackage{subfigure}														%% 
\usepackage{lscape}															%% Ermöglicht gedrehte Bilder



%%% c) Tabellen
\usepackage{multirow}                            							%% Tabellen
\usepackage{booktabs}														%% Ermöglicht die Gestaltung von horizontalen linien in Tabellen
\usepackage{dcolumn}														%% Ermöglicht Gestaltung von Tabellen
\usepackage{tabularx}														%% Gestaltung von Tabellen; u.a. feste Zeilenbreite und automatischer Zeilenumbruch in Tabellen



%%% d) Formatierung
\usepackage[hang,bf, labelformat=simple]{caption} 		                						%% Formatierung von Abbildungs- und Tabellenbeschriftungen
\usepackage{capt-of}														%% Ermöglicht das Erstellen einer Caption
\usepackage{color}                             				  				%% Textfarbe
\usepackage{url} 															%% URLs einfügen
\usepackage[pdfpagelabels]{hyperref} 										%% pdf-Optionen (sehr gutes Paket für Verwendung des pdf-Dokuments)
\usepackage{psfrag}															%% Beschriftung von Abbildungen durch Ersetzen von Platzhaltern
\usepackage{paralist}														%% Ermöglicht kompakte Aufzählungen
\usepackage{array}															%% Ermöglicht eine genauere Positionierung von Objekten
\usepackage{rotating}														%% Drehen von Objekten; z.B. Bilder inkl. Beschriftung
\usepackage{shortvrb}														%% kürzere Form der \verb Funktion die es ermöglicht Latexbefehle in bestimmten Bereichen zu ignorieren
\usepackage{xcolor}															%% Zum einfärben von minipages mittels (Bsp.:) \fcolorbox{red}{gray}{...}
\usepackage{icomma}															%% Keine Lücke hinter Komma in Matheumgebung


%%% e) Sonstige
\usepackage[numbers]{natbib}												%% Einstellungen für Bibtex
\usepackage{import}															%% Notwendig um Unterordner mit \import für *.pdf_tex (aus InkScape) Dateien aufzurufen
\usepackage[textsize=scriptsize, german]{todonotes}							%% Ermöglicht des erstellen von Notizen. Mit [..., disable] werden alle Notizen ausgeblendet. \listoftodos erstellt eine Übersicht der todos
\usepackage{pdfpages}														%% 
\usepackage{makeidx}	%% 

\usepackage[skip=0pt]{caption}
\usepackage{placeins} % put this in your pre-amble
\usepackage{flafter}  % put this in your pre-amble
\usepackage{annotate-equations}
\usepackage{mathtools}
\usepackage[makeroom]{cancel}
\usepackage{esvect}
\usepackage{graphicx}

