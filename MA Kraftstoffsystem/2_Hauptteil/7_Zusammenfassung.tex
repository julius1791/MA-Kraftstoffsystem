%==============================================================================
\chapter{Zusammenfassung und Ausblick}
\label{chap:fazit}
%==============================================================================

Der Einsatz wasserstoffbetriebener Fluggasturbinen erfordert Leistungsfähige Kraftstoffsysteme, die den erheblichen Wärmebedarf für die Vorkonditionierung des flüssigen Wasserstoffs decken können. Kenntnis der spezifischen Wärme- und Leistungsanforderungen in den relevanten Betriebspunkten ist dabei essenziell für die angemessene Dimensionierung der Systemkomponenten. Der Fokus dieser Arbeit ist die Entwicklung einer Modellierung zur Abschätzung dieser Anforderungen, insbesondere in Abhängigkeit der durch die Systemauslegung zu bestimmenden Eintrittstemperaturen. 

Im Rahmen dieser Arbeit wurden ein Referenzkraftstoffsystem und drei Wasserstoff-Kraftstoffsysteme entwickelt. Die Modellierung der Kraftstoffsysteme erfolgte anhand von Komponenten- und Stoffmodellen, die auf Basis einer umfassenden Literaturrecherche für den Betriebspunkt eines Verkehrsflugzeugs im Reiseflug parametriert wurden. Diese Modellierung bestimmt den Wärmebedarf, die erforderlichen Pumpen- und Verdichterleistungen sowie den daraus resultierenden Mehrverbrauch in Abhängigkeit von den Auslegungsgrößen.  Zudem ermöglichen die zugrunde liegenden Komponenten- und Stoffmodelle eine flexible Kombination in verschiedenen Anordnungen, um alternative Kraftstoffsystem-Konzepte abzubilden und zu vergleichen.

Die Methodik dieser Arbeit wurde auf ein bekanntes Kraftstoffsystem aus der Literatur angewandt, um die Aussagekraft des Ansatzes zu bestätigen. Das Ergebnis dieser Validierung ergab eine Übereinstimmung mit den Literaturdaten, jedoch nur in Teilen. Der Ursprung dieser Abweichung konnte nicht abschließend geklärt werden. Im Rahmen einer zweidimensionalen Parameterstudie der Wasserstoff-Kraftstoffsysteme konnte ein Zusammenhang zwischen Kraftstoffverbrauch und Wärmeübertrager-Eintrittstemperatur festgestellt werden. In dem Anschließenden Vergleich mit dem Referenzkraftstoffsystem konnte gezeigt werden, dass die Wasserstoff-Kraftstoffsysteme im Reiseflug trotz höherem Wärmebedarf und höherer notwendiger Leistungsentnahme einen geringeren Energieverbrauch ermöglichen. 

Zukünftige Arbeiten könnten sich mit abweichenden Anordnungen der Komponenten der Wasserstoff-Kraftstoffsysteme befassen, insbesondere im Hinblick auf die Integration von Wärmeübertragern und der Rezirkulation, um die Wärmeübertrager-Eintrittstemperatur zu erreichen. Fortgeschrittene Arbeiten könnten sich damit beschäftigen die Modellierung auf Gesamttriebwerksebene in Leistungsrechnungen zu integrieren und die Modellierung durch die Berücksichtigung von transienten und Off-Design-Punkten zu erweitern. Dies würde zudem die Entwicklung einer Regelstrategie für die einzelnen Kraftstoffsystemkomponenten erfordern.